\thispagestyle{empty}

\begin{center}
{\large\bfseries Algoritmos Basados En Población Con Diversidad Mejorada \\ Algoritmo basado en Un Mundo Feliz de Aldous Huxley }\\
\end{center}
\begin{center}
Cecilia Merelo Molina\\
\end{center}

%\vspace{0.7cm}

\vspace{0.5cm}
\noindent{\textbf{Palabras clave}: \textit{software libre}, \textit{open source}, \textit{castas}, \textit{Un mundo feliz},
\textit{diversidad}, \textit{ágil}, \textit{metáfora}, \textit{algoritmo genético}, \textit{metaheurística}
\vspace{0.7cm}

\noindent{\textbf{Resumen}\\

Al principio de este año leí el libro de Aldous Huxley ``Un Mundo Feliz``. Esta novela describe una distopía,
que anticipa el desarrollo en tecnología reproductiva, y describe cómo con esta tecnología se ha creado
la raza humana perfecta; esto es, básicamente un algoritmo de optimización basado en una población, es decir,
un algoritmo evolutivo. El libro describe el proceso para crear el ``humano perfecto``, este símil 
será el que se tratará de reflejar en el TFG. El objetivo es desarrollar un algoritmo genético basado en el proceso de
fecundación y división en castas descrito en el libro, y compararlo con otros algoritmos para ver cómo se comporta,
analizando cómo la división en castas afecta a la diversidad de la población.

\cleardoublepage

\begin{center}
	{\large\bfseries Algorithm Based In a Diversity-Improved Population}\\
\end{center}
\begin{center}
	Cecilia Merelo Molina\\
\end{center}
\vspace{0.5cm}
\noindent{\textbf{Keywords}: \textit{open source}, \textit{castes}, \textit{A brave new world},
\textit{diversity}, \textit{agile}, \textit{metaphor}, \textit{genetic algorithms}, \textit{metaheuristics}
\vspace{0.7cm}

\noindent{\textbf{Abstract}\\

At the beginning of this year I read "A brave new world" by Aldous Huxley. This novel describes a dystopia, 
which anticipates the development of breeding technology, and how this technology creates the perfect human race; 
that is, essentially a population based optimization algorithm, of which a well known kind are evolutionary algorithms. 
This book kind of describes the process for making the “perfect human”, so this similitude is what we will try to develop in this paper. 
The goal is to develop a genetic algorithm based on the fecundation process of the book and compare it to other 
algorithms to see how it behaves, investigating how the division in castes affects the diversity in the population.

\cleardoublepage

\thispagestyle{empty}

\noindent\rule[-1ex]{\textwidth}{2pt}\\[4.5ex]

D. \textbf{Juan Julián Merelo Guervós}, Profesor(a) del departamento de Arquitectura y Tecnología de Computadores.

\vspace{0.5cm}

\textbf{Informo:}

\vspace{0.5cm}

Que el presente trabajo, titulado \textit{\textbf{Algoritmos Basados En Población Con Diversidad Mejorada}},
ha sido realizado bajo mi supervisión por \textbf{Cecilia Merelo Molina}, y autorizo la defensa de dicho trabajo ante el tribunal
que corresponda.

\vspace{0.5cm}

Y para que conste, expiden y firman el presente informe en Granada a septiembre de 2021.

\vspace{1cm}

\textbf{El/la director(a)/es: }

\vspace{5cm}

\noindent \textbf{(Juan Julián Merelo Guervós)}

\chapter*{Agradecimientos}

Quiero agradecer a mi familia por todo el apoyo que me han dado en estos meses de trabajo. A todas las personas que
se lo han leído y me han añadido puntos y comas. A mi tutor por estar tan atento a los mensajes.  A todos los compañeros 
con los que he trabajo en Badger Maps por enseñarme todo lo que se sobre desarrollo ágil y calidad del código. A todas
mis amiguitas por haberme amenizado los días más duros y haberme animado durante todo el proceso. 

