\chapter{Descripción Del Problema}

En 2019, para la asignatura de metaheurísticas de la Universidad de Granada se me propuso inventarme una metaheuristica. En
ese momento me acababa de leer el libro "Un Munfo feliz" de Aldous Huxley, así que decidí adaptar el proceso de creación 
de humanos que decribía el libro a un algoritmo genético. A lo largo del desarrollo se me presentaron varios problemas. El \textit{primero} fue
la planificación y el desarrollo del algoritmo. Cuando consideré que había terminado el desarrollo no lo había ejecutado ni una vez. El
\textit{segundo problema} fue el lenguage escogido, lo desarrollé en Python, un lenguage caracterizado por su fácil sintaxis pero su deficiencia de
velocidad. Cada ejecución del problema llevaba horas, por lo que la experimentación que realice fue bastante pobre. Con esto llegamos
al \textit{tercer problema}: la optimización de los parámetros. En este trabajo se quieren abordar estos 3 problemas. 

A la hora del desarrollo se priorizarán conceptos como arquitectura limpia \cite{cleanArquitecture2017} y la limpieza del
código, así como el desarrollo basado en tests. Para asegurar que una vez tengamos el algoritmo completo está cubierto de tests y, por tanto, sabemos que funciona. Además
se seguirá el método Kanban, es decir, el proceso de desarollo se realizará en iteraciones pequeás, que permitirán tener un algoritmo que se pueda probar
al final de cada una. 

Esta vez el código está escrito en Julia, un lenguage de programación multiparadigma de tipado dinámico. Orientado a la computación técnica y 
científica. La rapidez fue uno de los principales objetivos a la de la creación del lenaguage. Julia es es miembro de el \"Petaflop Club\", que incluye 
aquellos lenguages que superan 1 petaflop/segundo como rendimiento pico. 

El análisis de el algoritmo y su comportamiento será clave en el trabajo, se quiere ver cómo la división en castas afecta a la diversidad.

\section{Naturaleza del algoritmo}

Estamos hablando de un algoritmo basado en la evolución de una población, así que seguirá la estructura de los
algoritmos evolutivos. Siguiendo la definición dada por Goldberg \cite{goldberg89}, "los Algoritmos Genéticos son algoritmos de búsqueda
basados en la mecánica de selección natural. Combinan la supervivencia del más apto entre estructuras de secuencias con un intercambio de 
información estructurado, aunque aleatorizado, para construir así un algoritmo
de búsqueda que tenga algo de las genialidades de las búsquedas humanas".

Para alcanzar la solución a un problema se parte de un conjunto inicial de individuos, llamado \textit{población},
generado de manera aleatoria. Cada uno de estos individuos representa una posible solución al problema. Estos individuos
evolucionarán siguiendo el proceso propuesto en el libro de Aldous Huxley, y se adaptarán en mayor o menor medida,
tras el paso de cada generación, a la solución requerida.
