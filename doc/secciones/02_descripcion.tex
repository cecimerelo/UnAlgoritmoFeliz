\chapter{Descripción del problema y caracterización de la solución}

En este capítulo se describe el problema cuya solución se quiere abordar en este trabajo. Primero se analiza la naturaleza del algoritmo que
se quiere implementar, luego se describe la metaheurística que se va a desarrollar, para acabar con los casos de uso que se han ido
implementando hasta conseguir la funcionalidad completa.

\section{Descripción}

En 2019, para la asignatura de metaheurísticas de la Universidad de Granada \cite{merelo_molina_2021} se propuso inventar una 
metaheurística. En ese momento acababa de leer el libro ``Un Mundo feliz`` de Aldous Huxley \cite{un_mundo_feliz_libro}, así que
decidí adaptar el proceso de optimización de la población humana que describe el libro a un algoritmo evolutivo. A lo largo del
desarrollo se me presentaron varios retos. El primero fue el lenguaje escogido; lo desarrollé en Python, un lenguaje caracterizado
por su fácil sintaxis pero su deficiencia de velocidad. Cada ejecución del problema llevaba horas, por lo que se hacía muy difícil
de ejecutar y analizar. El segundo fue la optimización de los parámetros; el algoritmo quedaba estancado en óptimos locales.
Estos problemas son los que se quieren abordar en el desarrollo de este trabajo.

Como se ha mencionado en la sección anterior, y relacionado con los objetivos de desarrollar una herramienta de altas prestaciones,
se quiere desarrollar una metaheurística inspirada por un libro. Pero no solo interesa
el poder adaptar el proceso descrito en el libro a una metaheurística. También se busca que la metaheurística pueda ser
usada fácilmente por la comunidad, que su implementación sea extensible y mantenible, y además fácil de analizar, para completar el objetivo
desarrollo mediante metodología ágil. Para conseguirlo se desarrollará una biblioteca que sea adaptable al problema a resolver, donde los parámetros iniciales sean indicados por la persona que use el algoritmo; parámetros como la función de fitness,
tamaño de la población inicial, dimensión del cromosoma y rango de búsqueda

\section{Naturaleza del algoritmo}

Como se ha mencionado, el algoritmo está basado en el proceso de optimización de la población humana que describe el libro, por tanto,
estamos hablando de un algoritmo basado en la evolución de una población, así que seguirá
la estructura de los algoritmos evolutivos, más en concreto, un \emph{algoritmo genético}. Se trata de un algoritmo 
genético ya que es un algoritmo de optimización de búsqueda inspirado en los procesos de Evolución Natural y
Evolución Genética.

Siguiendo la definición dada por Goldberg \cite{goldberg89}, "los Algoritmos Genéticos son algoritmos de búsqueda
basados en la mecánica de selección natural. Combinan la supervivencia del más apto entre estructuras de secuencias con un intercambio de 
información estructurado, aunque aleatorizado, para construir así un algoritmo de búsqueda que tenga algo de las genialidades de las 
búsquedas humanas". En resumen, estos algoritmos reflejan el proceso de la selección natural. Forman parte del grupo de 
los \emph{algoritmos evolutivos}. El proceso está compuesto de 4 componentes principales:

\begin{itemize}
    \item Creación de la población inicial: el proceso comienza con un conjunto de individuos llamado \emph{Población}. Cada individuo
    representa una solución del problema a resolver. Cada individuo se caracteriza por una serie de parámetros llamados
    \emph{genes}. Los genes se unen para formar un \emph{Cromosoma}. Estos se crean en la primera generación. La población del 
    resto de generaciones serán evoluciones de esta población inicial.
    \item Selección: la idea de la fase de selección es seleccionar los individuos más adecuados para que pasen sus
    genes a las siguientes generaciones. Los individuos con mejor valor de fitness son los que tienen más posibilidades
    de ser seleccionados para la reproducción.
    \item Cruce: produce un nuevo individuo por cada pareja de padres, seleccionando de uno de ellos un segmento
    continuo de características y copiándolo en la descendencia. Los genes que quedan por asignar en la descendencia
    combinan de manera uniforme características de los dos padres.
    \item Mutación: en los hijos, algunos de los genes pueden ser sometidos a mutación, es decir, alguno de sus genes
    serán cambiados.
\end{itemize}

Además de las fases, también se debe definir una \emph{condición de parada} y una \emph{función de fitness}. La
condición de parada sirve para que el algoritmo finalice si la población converge, es decir, si no produce hijos 
que sean diferentes con respecto a las generaciones anteriores, o si alcanza la solución. La función de fitness
determina cómo de adecuado es un individuo, esto es, la probabilidad de que un individuo sea seleccionado para
la reproducción depende de este valor. Por tanto, para alcanzar la solución a un problema se parte de un conjunto 
inicial de individuos, llamado \textit{población inicial}, generado de manera aleatoria. Cada uno de estos 
individuos representa una posible solución al problema. Estos individuos van pasando por los diferentes operadores y 
evolucionando en cada generación.

En el caso del algoritmo propuesto en este trabajo (\emph{Algoritmo Feliz}) la población evolucionará siguiendo un proceso 
inspirado por los distintos mecanismos descritos en el libro de Aldous Huxley. El proceso propuesto se describe con mayor detalle
en la sección \ref{sect: description}. Al usar como guía los procesos descritos en el libro quiere decir que se está desarrollando 
una \textit{metaheurística}. Las  metaheurísticas son métodos de alto nivel, independientes del problema, que proporcionan una serie de guías o
estrategias para desarrollar algoritmos de optimización heurísticos \cite{metaheuristics_def}. Muchas metaheurísticas
están inspiradas en la naturaleza \cite{Molina2020ComprehensiveTO}. Por ejemplo, la optimización basada en colonias de
hormigas que imita la manera en la que las hormigas se comportan cuando viajan a una fuente de comida, y cómo se 
comunican unas con otras a través de feromonas. 

Normalmente las metaheurísticas se usan en problemas de optimización por varias razones:
\begin{itemize}
    \item Encuentran maneras de ir de una solución a otra mejor sin tener por qué considerar todas las posibles combinaciones. 
    \item Pueden evitar quedarse estancadas en mínimos locales porque usan aleatoriedad.
\end{itemize}

En conclusión, se va a desarrollar una metaheurística basada en el proceso de fecundación y la división en castas descrita en el 
libro de A. Huxley, llamada ``Algoritmo Feliz`` o \emph{AF}.

\section{Descripción de la metaheurística desarrollada} \label{sect: description}

El libro describe cómo se alcanza esta raza humana perfecta mediante una ``cadena de montaje``, con varias fases. Esto lo 
reflejaremos mediante un algoritmo evolutivo generacional con los operadores de selección, cruce, mutación y reemplazamiento. 
El proceso comienza en la \textbf{\textit{Sala de Fecundación}}; aquí se crean los óvulos y se fecundan. Una vez \textit{fecundados}
los óvulos pasan a las incubadoras, donde se decide la \textit{casta} a la que pertenecerá cada individuo. Huxley describe 
como a los Alfas y los Betas se les suministra una gran cantidad de nutrientes y hormonas durante la incubación. Mientras que las
castas bajas, Gamma, Delta y Epsilon son privadas de estos elementos necesarios para el desarrollo. Para imitar esta
 ``falta de nutrientes``, en el algoritmo a desarrollar privaremos a las castas bajas de operadores, solo mutarán. Con esto
 pretendo avanzar el objetivo de mantener la diversidad alta durante toda la ejecución del algoritmo. 
 Las castas se implementarán de la siguiente manera: 

\begin{itemize}
    \item \textbf{Alfas}:
        \begin{itemize}
            \item Libro: son los más inteligentes, a este grupo pertenece la élite. Tienen responsabilidades y son
            los que tienen la capacidad de tomar decisiones.
            \item Implementación: se reproducirán solo entre ellos, pasarán por todos los operadores del algoritmo.
        \end{itemize}
    \item \textbf{Betas}: 
        \begin{itemize}
            \item Libro: son menos inteligentes que los anteriores y su función principal se reduce a tareas
            administrativas.
            \item  Implementación: el cruce sólo se realiza con individuos Alpha.
        \end{itemize}
    \item \textbf{Gammas}: 
        \begin{itemize}
            \item Libro: son los empleados subalternos, cuyas tareas requieren de habilidad. Son expertos en tareas repetitivas
            \item Implementación: los individuos de esta casta solo tendrán mutación por búsqueda local.
        \end{itemize}
    \item \textbf{Castas Bajas}: sólo tendrán el operador de mutación por segmento fijo, sin búsqueda local. A este sector pertenecen:
        \begin{itemize}
            \item \textbf{Deltas}:a este grupo pertenecen los empleados de los anteriores.
            \item \textbf{Epsilones}: es la casta inferior, a ella pertenecen los empleados para trabajos arduos.
        \end{itemize}
\end{itemize}

Con esta estructura en mente la metaheurística se divide en las siguientes fases:

\begin{itemize}
    \item \textbf{Sala de Fecundación}: los individuos son creados de manera totalmente aleatoria. Tantos individuos
    como indique un parámetro \textit{I}.
    \item \textbf{Sala de incubación}: en esta fase realizamos la división en castas. Se hará basándonos en el valor de
    la función objetivo del individuo. Además cada casta tendrá un porcentaje diferente de la población. Ya que en el libro
    describe que las castas bajas se reproducen mediante el proceso de \textit{Bokanovsky}, donde un embrión, que normalmente 
    se desarrolla hasta convertirse en un adulto, se convierte en 96 gemelos idénticos. Esto en el algoritmo se reflejará 
    en el tamaño de la población, que descenderá cuanto más alta sea la casta.    
    \item \textbf{Evolución de las castas}: cada casta seguirá un proceso de evolución diferente como ya se 
    ha mencionado.
\end{itemize}

No se trata de castas estáticas, se generan al principio de cada generación. Imaginemos que tenemos una población de 10 individuos, cada 
uno con un valor de fitness. En la primera iteración se dividirá esta población atendiendo a este valor de fitness. Posteriormente,
cada individuo seguirá el proceso de evolución correspondiente a su casta. Al final de la generación se juntan todos los cromosomas,
independientemente de su casta, ya habrá evolucionado su valor con respecto al resto de los individuos de la
población. La siguiente generación comienza dividiendo en castas los cromosomas que han evolucionado en la anterior.

\section{Naturaleza del software} \label{sect:no-funcionales}

Alineados con los objetivos del TFG \ref{sect:goals}, los requisitos no funcionales de este proyecto son:

\begin{itemize}
    \item Desarrollo \emph{future-proof}: toda la funcionalidad debe ser testeada, de esta manera, en el futuro,
    cuando se hagan cambios se sabrá si la nueva funcionalidad rompe lo escrito hasta el momento o no. Esto
    también favorece que gente ajena al proyecto se vea más segura a participar en él. Además esto incrementa
    la mantenibilidad del código, ya que los tests ayudan a que siempre esté claro lo que hace cada método
    del código.
    \item Evitar la repetición de código, esto favorece a que el código sea más fácil de mantener.
    \item Desarrollo ágil: desarrollar la mínima cantidad de código que cubra la funcionalidad.
\end{itemize}

Este proyecto es software libre, y está liberado con la licencia \cite{gplv3}. El producto final de este proyecto será una
biblioteca que estará disponible Github \cite{project_repository}. Esto ayuda a la reproducibilidad de la ciencia y 
adicionalmente es parte de desarrollo ágil, ya que cualquier persona puede añadir tareas y marcarle la prioridad. Esto va
también en línea con la ciencia abierta. El concepto de “ciencia abierta” es la difusión del conocimiento científico 
lo más amplia posible, libre para todas y todos, accesible en línea y reutilizable
