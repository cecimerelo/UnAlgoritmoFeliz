\chapter{Introducción}

\section{Motivación}

A principios de este año leí el conocido libro de Aldous Huxley: Un mundo feliz. La novela es una
distopía que describe el desarrollo en tecnología reproductiva, cultivos humanos e hipnopedia y el manejo de las
emociones por medio de drogas. La población se ordena en \textbf{castas}, asignadas desde el nacimiento, donde cada uno
sabe y acepta su lugar en la sociedad. La guerra y la pobreza han sido erradicadas, y todos son permanentemente
felices.

Cuando hablamos de algoritmos evolutivos nuestro objetivo es alcanzar la solución óptima de un problema, y este libro describe el 
proceso mediante el cual han alcanzado la raza humana perfecta. Por tanto el objetivo final de este trabajo es desarrollar
un algoritmo evolutivo basándonos en el proceso de fecundación del libro y comparar su comportamiento con otros algoritmos. Además
investiga cómo la división en castas afecta la diversidad en la población.

\section{Objetivos del trabajo}

\begin{enumerate}
    \item Adaptar la división en castas y el concepto de raza humana perfecta descrita en el libro a un algoritmo evolutivo.
    \item Investigar cómo afecta la división en castas a la diversidad.
    \item Desarrollar el algoritmo siguiendo metodologías ágiles
    \item Demostrar que el lenguaje de programación Julia es buena opción para el desarrollo de algoritmos
\end{enumerate}

\section{Conceptos generales}

En el libro se describe cómo se alcanza esta raza humana perfecta mediante una \"cadena de montaje\", con
varias fases. Esto lo reflejaremos mediante un algoritmo evolutivo generacional con los operadores de selección, cruce, mutación
y reemplazamiento. El proceso comienza en la \textbf{\textit{Sala de Fecundación}}, aquí se crean los óvulos y se fecundan. Una
vez \textit{fecundados} los óvulos pasan a las incubadoras, donde se decide la \textit{casta} a la que pertenecerá cada individuo:

\begin{itemize}
    \item \textbf{Alphas}:
        \begin{itemize}
            \item Libro: son los más inteligentes, a este grupo pertenece la élite. Tienen responsabilidades y son
            los que tienen la capacidad de tomar decisiones.
            \item Implementación: se reproducirán sólo entre ellos, pasarán por todos los operadores del algoritmo.
        \end{itemize}
    \item \textbf{Betas}: 
        \begin{itemize}
            \item Libro: son menos inteligentes que los anteriores y su función principal se reduce a tareas
            administrativas.
            \item  Implementación: el cruce sólo se realiza con individuos Alpha
        \end{itemize}
    \item \textbf{Gammas}: 
        \begin{itemize}
            \item Libro: son los empleados subalternos, cuyas tareas requieren de habilidad. Son expertos en tareas repetitivas
            \item Implementación: los individuos de esta casta solo tendrán mutación, por búsqueda local.
        \end{itemize}
    \item \textbf{Castas Bajas}: sólo tendrán el operador de mutación por segmento fijo, sin búsqueda local. A este sector pertenecen:
        \begin{itemize}
            \item \textbf{Deltas}:a este grupo pertenecen los empleados de los anteriores.
            \item \textbf{Epsilones}: es la casta inferior, a ella pertenecen los empleados para trabajos arduos.
        \end{itemize}
\end{itemize}

Con esta estructura en mente la metaheurística se divide en las siguientes fases:

\begin{itemize}
    \item \textbf{Sala de Fecundación}: los individuos son creados de manera totalmente aleatoria. Tantos individuos
    como indique un parámetro \textit{I}.
    \item \textbf{Sala de incubación}: en esta fase realizamos la división en castas. Se hará basándonos en el valor de
    la función objetivo del individuo. Además cada casta tendrá un porcentaje diferente de la población.
    \item \textbf{Evolución de las castas}: cada casta seguirá un proceso de evolución diferente como ya se 
    ha mencionado.
\end{itemize}

El tamaño de la población descenderá cuanto más alta sea la casta.


Este proyecto es software libre, y está liberado con la licencia \cite{gplv3}.