\chapter{Introducción}

\section{Motivación}

A principios de este año leí el conocido libro de Aldous Huxley: Un mundo feliz. La novela es una
distopía que describe el desarrollo en tecnología reproductiva, cultivos humanos e hipnopedia y el manejo de las
emociones por medio de drogas. La población se ordena en \textbf{castas}, asignadas desde el nacimiento, donde cada uno
sabe y acepta su lugar en la sociedad. La guerra y la pobreza han sido erradicadas, y todos son permanentemente
felices. Se trata de un ``mundo óptimo``', cuya optimización está basada en la población y en el equilibrio que crea
la división en castas, no en un único individuo 

Cuando hablamos de algoritmos evolutivos nuestro objetivo es alcanzar la solución óptima de un problema, y este libro describe el 
proceso mediante el cual han alcanzado la raza humana perfecta. Por tanto se quiere desarrollar un algoritmo evolutivo basado en 
el proceso de fecundación del libro y comparar su comportamiento con otros algoritmos. Además investigar cómo la división en castas afecta 
la diversidad en la población.

\section{Conceptos generales}

El libro describe cómo se alcanza esta raza humana perfecta mediante una ``cadena de montaje``, con varias fases. Esto lo reflejaremos mediante 
un algoritmo evolutivo generacional con los operadores de selección, cruce, mutación y reemplazamiento. El proceso comienza en la \textbf{\textit{Sala de Fecundación}}, 
aquí se crean los óvulos y se fecundan. Una vez \textit{fecundados} los óvulos pasan a las incubadoras, donde se decide la \textit{casta} a la que pertenecerá cada individuo. Huxley
describe como a los Alfas y los Betas se les suministra una gran cantidad de nutrientes y hormonas durante la incubación. Mientras que a las castas bajas, Gamma, Delta y
Epsilon son privados de estos elementos necesarios para el desarrollo. Para imitar esta ``falta de nutrientes``', en el algoritmo a desarrollar privaremos a las castas bajas de
operadores, solo mutarán. Las castas se implementarán de la siguiente manera: 

\begin{itemize}
    \item \textbf{Alfas}:
        \begin{itemize}
            \item Libro: son los más inteligentes, a este grupo pertenece la élite. Tienen responsabilidades y son
            los que tienen la capacidad de tomar decisiones.
            \item Implementación: se reproducirán solo entre ellos, pasarán por todos los operadores del algoritmo.
        \end{itemize}
    \item \textbf{Betas}: 
        \begin{itemize}
            \item Libro: son menos inteligentes que los anteriores y su función principal se reduce a tareas
            administrativas.
            \item  Implementación: el cruce sólo se realiza con individuos Alpha.
        \end{itemize}
    \item \textbf{Gammas}: 
        \begin{itemize}
            \item Libro: son los empleados subalternos, cuyas tareas requieren de habilidad. Son expertos en tareas repetitivas
            \item Implementación: los individuos de esta casta solo tendrán mutación, por búsqueda local.
        \end{itemize}
    \item \textbf{Castas Bajas}: sólo tendrán el operador de mutación por segmento fijo, sin búsqueda local. A este sector pertenecen:
        \begin{itemize}
            \item \textbf{Deltas}:a este grupo pertenecen los empleados de los anteriores.
            \item \textbf{Epsilones}: es la casta inferior, a ella pertenecen los empleados para trabajos arduos.
        \end{itemize}
\end{itemize}

Con esta estructura en mente la metaheurística se divide en las siguientes fases:

\begin{itemize}
    \item \textbf{Sala de Fecundación}: los individuos son creados de manera totalmente aleatoria. Tantos individuos
    como indique un parámetro \textit{I}.
    \item \textbf{Sala de incubación}: en esta fase realizamos la división en castas. Se hará basándonos en el valor de
    la función objetivo del individuo. Además cada casta tendrá un porcentaje diferente de la población.
    \item \textbf{Evolución de las castas}: cada casta seguirá un proceso de evolución diferente como ya se 
    ha mencionado.
\end{itemize}

Por último, el libro describe que las castas bajas se reproducen mediante el proceso de \textit{Bokanovsky}, donde un embrión, que normalmente se desarrolla hasta convertirse
en un adulto, se convierte en 96 gemelos idénticos. Esto en el algoritmo se reflejará en el tamaño de la población, que descenderá cuanto más alta sea la casta.


Este proyecto es software libre, y está liberado con la licencia \cite{gplv3}.

\section{Objetivos del trabajo}

\begin{enumerate}
    \item Mejorar la diversidad en problemas de optimización mediante la división de la población en castas.
    \item Desarrollar una herramienta de altas prestaciones para la ejecución de algoritmos evolutivos.
\end{enumerate}