\chapter{Implementación}
En esta sección se va a describir la implementación, basada en los requisitos mencionados.

\section{Lenguaje de programación}

Una de las formas de conseguir altas prestaciones, para alcanzar el objetivo 2, es escoger el lenguaje adecuado. Por ello se ha escogido
un lenguaje que hace poco entró en el \"Petaflop Club\", que incluye aquellos lenguajes que superan 1 petaflop/segundo como rendimiento pico. Se trata
de Julia \cite{julia}, un lenguaje de programación multiparadigma de tipado dinámico. Orientado a la computación técnica y
científica. El objetivo de sus creadores fue el de combinar lo mejor de otros lenguajes, es decir, conseguir un “todo en uno”, lo que se traduce en:
la velocidad de C, la usabilidad de Python, el dinamismo de Ruby, la destreza matemática de MatLab, y las estadísticas de R \cite{julia_goals}.

\section{Metodología de desarrollo}

Un requisito fundamental a la hora de implementar es usar una metodología que garantice la evolución del código. Por eso la implementación 
del software se ha dividido en tareas siguiendo la metodología ágil Kanban. Se divide el proyecto en tareas pequeñas que aporten valor al producto 
final. De esta forma puedes probar cada tarea a medida que la vas acabando. Al final del desarrollo estás segura de que todas las piezas funcionan
correctamente.

Otro requisito es desarrollar un algoritmo que sea mantenible. Por tanto, a la hora del desarrollo se priorizarán conceptos 
como arquitectura limpia \cite{cleanArquitecture2017} y la limpieza del código \cite{cleanCode2008}, así como el desarrollo basado en tests. De esta manera, se asegura
que el algoritmo está asentado sobre una base firme. Asimismo, gracias al desarrollo basado en tests para cada cambio que se haga
se sabrá si se ha roto lo que estaba escrito hasta el momento.  Usar estándares va en línea con el objetivo de aplicar el desarrollo ágil a la ciencia.

\section{Estructuras de datos}

Las entidades que se han utilizado para estructura la información son:
\begin{itemize}
    \item Para tener bien definidos los datos de entrada se ha creado una entidad que contiene todos los parámetros de
    configuración del algoritmo. Se trata de una estructura inmutable, estos parámetros no cambiarán a lo largo de todo el
    algoritmo.
    \item Para unificar todos los datos que necesita el algoritmo, se ha definido una estructura inmutable. Una vez
    rellenada esta entidad, el algoritmo tiene todos los datos necesarios para comenzar la ejecución.
    \item Definir la casta como una entidad abstracta. Julia usa el \emph{dispatch} múltiple como paradigma, 
    esto va a ser muy útil a la hora de definir los operadores como se verá más adelante. Teniendo la casta 
    como tipo abstracto podremos definir métodos que acepten cualquier tipo de casta.

    \item Julia es un lenguaje de tipado dinámico, pero tiene alguna de las ventajas que del tipado estático haciendo posible indicar
    los tipos de algunos valores. Además, Para poder hacer buen uso de este paradigma se han definido las castas como tipos distintos, por ejemplo, para
    la casta alfa:

    \begin{lstlisting}[
        basicstyle=\small
    ]
        @with_kw struct ALPHA <: Caste
            name::String = "ALPHA"
        end
    \end{lstlisting}

    El decorador \lstinline{with_kw} de la biblioteca \emph{Parameters.jl}, para poder añadir parámetros por defecto a una entidad. Las castas se han definido 
    como tipos inmutables, ya que en ningún momento se va a cambiar el nombre propio de la casta.
\end{itemize}

\section{Implementación de los casos de uso}

En esta sección se va a describir las decisiones técnicas que se han tomado para desarrollar los casos de uso \ref{sect:use_cases}

\begin{itemize}
    \item \textbf{Sistema de entrada de la biblioteca} \ref{it:sist_entrada}: usar el formato JSON para definir los parámetros
    del algoritmo. Un ejemplo de fichero sería el siguiente:
    \begin{lstlisting}[
        basicstyle=\small
    ]
        {
            "CHROMOSOME_SIZE": 2,
            "POPULATION_SIZE": 100,
            "MAX_GENERATIONS": 15,
            "POPULATION_PERCENTAGE": {
                "ALPHA": 10,
                "BETA": 20,
                "GAMMA": 20,
                "DELTA": 20,
                "EPSILON": 30
            },
            "MUTATION_RATE": {
                "ALPHA": 10,
                "BETA": 10,
                "GAMMA": 10,
                "DELTA": 10,
                "EPSILON": 10
            }
        }
    \end{lstlisting}

    Cualquiera de estos parámetros es se puede cambiar. Esto se convertirá en una entidad inmutable que servirá como 
    entrada al algoritmo.

    \item \textbf{Función de fitness \cite{project_repository_8}}: se ha usado las funciones que pertenecen al \emph{benchmark}
    \emph{ Black-box Optimization Benchmarking (BBOB)} \cite{bbob_definition}. Este \emph{benchmark} define funciones que se suelen 
    usar como referencia en el campo de las metaheurísticas. En Julia están implementadas en la biblioteca 
    \emph{BlackBoxOptimizationBenchmarking.jl} \cite{bbob_jl}.

    \item \textbf{Operadores de selección, cruce y mutación} [\ref{it:selection_operator}, \ref{it:crossover_operator}, \ref{it:mutation_operator}]:
    Se ha usado el \emph{dispatch} múltiple para definir operadores que se adecúen en cada casta. Por ejemplo, para el caso del
    operador de selección se han definido 3 tipos de parámetros de entrada:

    \begin{lstlisting}[
        basicstyle=\small
    ]
        selector_operator(
            caste::ALPHA, 
            caste_population
        )
        selector_operator(
            caste::BETA, 
            caste_population, 
            alpha_reproduction_pool
        )
        selector_operator(
            caste, 
            caste_population
        )
    \end{lstlisting}
    
    Para muestrear la población se ha usado la biblioteca \emph{StatsBase.jl}. Se ha elegido esta librería porque
    tiene un gran número de contribuidores y está en continua actualización
\end{itemize}

\section{Implementación del cálculo de la diversidad}

Uno de los objetivos del TFG es mejorar la diversidad en métodos que solucionan problemas de optimización mediante la división de la población en castas.
Para calcularla se han usado como métricas la entropía y la distancia de edición, la decisión de usar estas medidas
está definida en secciones posteriores. Para calcular la entropía en Julia hay varias librerías disponibles:

\begin{itemize}
    \item \emph{Entropy.jl} \cite{entropy_jl}: este paquete contiene funcionalidad para ejecutar estimaciones de la entropía. No se ha usado porque,
    aparte de no tener nada de documentación, la última vez que se actualizó fue hace 8 años, por tanto contiene
    sintaxis antigua que no compilará con las nuevas versiones del lenguaje.
    \item \emph{Diversity.jl} \cite{diversity_jl}: este paquete provee de funcionalidad para calcular diferentes
    tipos de diversidad. Es una biblioteca muy completa, pero tiene una curva de aprendizaje muy grande, ya que los
    operadores definidos para el cálculo son muy teóricos. Además no tiene para calcular la entropía y la distancia
    de edición.
\end{itemize}

Al final para calcular la entropía se ha usado el paquete \emph{InformationMeasures.jl} \cite{informationMeasures_jl}, ya 
que tiene una función que te calcula la entropía para cualquier tipo de datos. Además ha sido muy fácil de usarla porque 
tiene buena documentación. La distancia de edición se calcula a partir de la distancia euclídea ofrecida por el paquete
\emph{Distances.jl} \cite{distances_jl}, paquete que ofrece gran cantidad de opciones para calcular distancias entre vectores.
Para calcular distancias era una de las pocas opciones disponibles, pero tiene muchos contribuidores, ya que forma 
parte de paquetes Julia Statistics \cite{stats_jl}.

\section{Implementación necesaria para el análisis de los resultados}

Para el análisis de los resultados se ha usado la estructura de datos \emph{DataFrames}, del paquete \emph{Dataframes.jl}. Se
ha usado esta estructura de datos porque facilita mucho la visualización en tablas. Además tiene una fuerte
integración con distintos formatos de salida de archivos. 

Para las gráficas se usará \emph{Gadfly.jl} \cite{gadfly_jl},ya que se integra estrechamente con los DataFrames
