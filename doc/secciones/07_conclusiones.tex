\chapter{Conclusiones y trabajo futuro}

Al principio de este trabajo se definieron 3 objetivos a alcanzar en el TFG. En esta sección
se analizarán para ver si se han llegado a completar. Además se comentarán cuáles serán
los siguientes pasos de este algoritmo.

\section{Conclusión}
El primer objetivo era mejorar la diversidad en problemas de optimización mediante la división de la población en castas.
En los últimos experimentos se ha visto como ajustando los parámetros se ha conseguido que el algoritmo no se quede estancado
en óptimos locales. Para poder afirmar si mejora la diversidad habría que hacer una comparación más precisa con el panorama actual 
de metaheurísticas. Pero para este objetivo, se puede concluir que con una buena elección de los parámetros iniciales se 
puede llegar a mantener una población más diversa que un algoritmo genético básico.

El segundo objetivo era desarrollar una herramienta de altas prestaciones para la ejecución de algoritmos evolutivos. En comparación
con la primera versión del algoritmo desarrollada en Python se puede confirmar que esta versión es mucho más rápida. En Python tardaba alrededor
de horas en ejecutarse, mientras que en esta nueva versión en Julia no se llega al minuto. Además se ha desarrollado una biblioteca mantenible,
ya que la funcionalidad está testeada al completo, lo que la protege ante cambios.

El tercer objetivo era aplicar el desarrollo ágil en la ciencia. Durante todo el trabajo se ha tenido una mentalidad ágil, preparada para adaptarse ante 
cualquier cambio de planificación, desarrollando en iteraciones que aportaban valor inmediato al producto.

\section{Trabajo futuro}

En un futuro sería interesante comprobar cómo se comporta el algoritmo al aplicarle paralelismo o investigar la comunicación entre las castas. 

Este trabajo ha sido presentado en ``International Seminar on Computational Intelligence 2021 (ISCI'21)``. Además será publicado 
en la recopilación de papers de este seminario.